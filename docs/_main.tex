% Options for packages loaded elsewhere
\PassOptionsToPackage{unicode}{hyperref}
\PassOptionsToPackage{hyphens}{url}
%
\documentclass[
]{book}
\usepackage{amsmath,amssymb}
\usepackage{iftex}
\ifPDFTeX
  \usepackage[T1]{fontenc}
  \usepackage[utf8]{inputenc}
  \usepackage{textcomp} % provide euro and other symbols
\else % if luatex or xetex
  \usepackage{unicode-math} % this also loads fontspec
  \defaultfontfeatures{Scale=MatchLowercase}
  \defaultfontfeatures[\rmfamily]{Ligatures=TeX,Scale=1}
\fi
\usepackage{lmodern}
\ifPDFTeX\else
  % xetex/luatex font selection
\fi
% Use upquote if available, for straight quotes in verbatim environments
\IfFileExists{upquote.sty}{\usepackage{upquote}}{}
\IfFileExists{microtype.sty}{% use microtype if available
  \usepackage[]{microtype}
  \UseMicrotypeSet[protrusion]{basicmath} % disable protrusion for tt fonts
}{}
\makeatletter
\@ifundefined{KOMAClassName}{% if non-KOMA class
  \IfFileExists{parskip.sty}{%
    \usepackage{parskip}
  }{% else
    \setlength{\parindent}{0pt}
    \setlength{\parskip}{6pt plus 2pt minus 1pt}}
}{% if KOMA class
  \KOMAoptions{parskip=half}}
\makeatother
\usepackage{xcolor}
\usepackage{color}
\usepackage{fancyvrb}
\newcommand{\VerbBar}{|}
\newcommand{\VERB}{\Verb[commandchars=\\\{\}]}
\DefineVerbatimEnvironment{Highlighting}{Verbatim}{commandchars=\\\{\}}
% Add ',fontsize=\small' for more characters per line
\usepackage{framed}
\definecolor{shadecolor}{RGB}{248,248,248}
\newenvironment{Shaded}{\begin{snugshade}}{\end{snugshade}}
\newcommand{\AlertTok}[1]{\textcolor[rgb]{0.94,0.16,0.16}{#1}}
\newcommand{\AnnotationTok}[1]{\textcolor[rgb]{0.56,0.35,0.01}{\textbf{\textit{#1}}}}
\newcommand{\AttributeTok}[1]{\textcolor[rgb]{0.13,0.29,0.53}{#1}}
\newcommand{\BaseNTok}[1]{\textcolor[rgb]{0.00,0.00,0.81}{#1}}
\newcommand{\BuiltInTok}[1]{#1}
\newcommand{\CharTok}[1]{\textcolor[rgb]{0.31,0.60,0.02}{#1}}
\newcommand{\CommentTok}[1]{\textcolor[rgb]{0.56,0.35,0.01}{\textit{#1}}}
\newcommand{\CommentVarTok}[1]{\textcolor[rgb]{0.56,0.35,0.01}{\textbf{\textit{#1}}}}
\newcommand{\ConstantTok}[1]{\textcolor[rgb]{0.56,0.35,0.01}{#1}}
\newcommand{\ControlFlowTok}[1]{\textcolor[rgb]{0.13,0.29,0.53}{\textbf{#1}}}
\newcommand{\DataTypeTok}[1]{\textcolor[rgb]{0.13,0.29,0.53}{#1}}
\newcommand{\DecValTok}[1]{\textcolor[rgb]{0.00,0.00,0.81}{#1}}
\newcommand{\DocumentationTok}[1]{\textcolor[rgb]{0.56,0.35,0.01}{\textbf{\textit{#1}}}}
\newcommand{\ErrorTok}[1]{\textcolor[rgb]{0.64,0.00,0.00}{\textbf{#1}}}
\newcommand{\ExtensionTok}[1]{#1}
\newcommand{\FloatTok}[1]{\textcolor[rgb]{0.00,0.00,0.81}{#1}}
\newcommand{\FunctionTok}[1]{\textcolor[rgb]{0.13,0.29,0.53}{\textbf{#1}}}
\newcommand{\ImportTok}[1]{#1}
\newcommand{\InformationTok}[1]{\textcolor[rgb]{0.56,0.35,0.01}{\textbf{\textit{#1}}}}
\newcommand{\KeywordTok}[1]{\textcolor[rgb]{0.13,0.29,0.53}{\textbf{#1}}}
\newcommand{\NormalTok}[1]{#1}
\newcommand{\OperatorTok}[1]{\textcolor[rgb]{0.81,0.36,0.00}{\textbf{#1}}}
\newcommand{\OtherTok}[1]{\textcolor[rgb]{0.56,0.35,0.01}{#1}}
\newcommand{\PreprocessorTok}[1]{\textcolor[rgb]{0.56,0.35,0.01}{\textit{#1}}}
\newcommand{\RegionMarkerTok}[1]{#1}
\newcommand{\SpecialCharTok}[1]{\textcolor[rgb]{0.81,0.36,0.00}{\textbf{#1}}}
\newcommand{\SpecialStringTok}[1]{\textcolor[rgb]{0.31,0.60,0.02}{#1}}
\newcommand{\StringTok}[1]{\textcolor[rgb]{0.31,0.60,0.02}{#1}}
\newcommand{\VariableTok}[1]{\textcolor[rgb]{0.00,0.00,0.00}{#1}}
\newcommand{\VerbatimStringTok}[1]{\textcolor[rgb]{0.31,0.60,0.02}{#1}}
\newcommand{\WarningTok}[1]{\textcolor[rgb]{0.56,0.35,0.01}{\textbf{\textit{#1}}}}
\usepackage{longtable,booktabs,array}
\usepackage{calc} % for calculating minipage widths
% Correct order of tables after \paragraph or \subparagraph
\usepackage{etoolbox}
\makeatletter
\patchcmd\longtable{\par}{\if@noskipsec\mbox{}\fi\par}{}{}
\makeatother
% Allow footnotes in longtable head/foot
\IfFileExists{footnotehyper.sty}{\usepackage{footnotehyper}}{\usepackage{footnote}}
\makesavenoteenv{longtable}
\usepackage{graphicx}
\makeatletter
\def\maxwidth{\ifdim\Gin@nat@width>\linewidth\linewidth\else\Gin@nat@width\fi}
\def\maxheight{\ifdim\Gin@nat@height>\textheight\textheight\else\Gin@nat@height\fi}
\makeatother
% Scale images if necessary, so that they will not overflow the page
% margins by default, and it is still possible to overwrite the defaults
% using explicit options in \includegraphics[width, height, ...]{}
\setkeys{Gin}{width=\maxwidth,height=\maxheight,keepaspectratio}
% Set default figure placement to htbp
\makeatletter
\def\fps@figure{htbp}
\makeatother
\setlength{\emergencystretch}{3em} % prevent overfull lines
\providecommand{\tightlist}{%
  \setlength{\itemsep}{0pt}\setlength{\parskip}{0pt}}
\setcounter{secnumdepth}{5}
\usepackage{booktabs}

\usepackage{color}
\usepackage{framed}
\setlength{\fboxsep}{.8em}

% These colours were manually entered, they shouldn't matter unless you want pdf output

\newenvironment{redbox}{
  \definecolor{shadecolor}{RGB}{243, 154, 157}
  \color{white}
  \begin{shaded}}
 {\end{shaded}}

\newenvironment{bluebox}{
  \definecolor{shadecolor}{RGB}{172, 210, 237}
  \color{white}
  \begin{shaded}}
 {\end{shaded}}

\newenvironment{greenbox}{
  \definecolor{shadecolor}{RGB}{141, 181, 128}
  \color{white}
  \begin{shaded}}
 {\end{shaded}}
\ifLuaTeX
  \usepackage{selnolig}  % disable illegal ligatures
\fi
\usepackage[]{natbib}
\bibliographystyle{plainnat}
\usepackage{bookmark}
\IfFileExists{xurl.sty}{\usepackage{xurl}}{} % add URL line breaks if available
\urlstyle{same}
\hypersetup{
  pdftitle={Introduction to R 2025},
  pdfauthor={Faculty: Mohamed Helmy},
  hidelinks,
  pdfcreator={LaTeX via pandoc}}

\title{Introduction to R 2025}
\author{Faculty: Mohamed Helmy}
\date{October 6-7, 2025}

\begin{document}
\maketitle

{
\setcounter{tocdepth}{1}
\tableofcontents
}
\part{Introduction}\label{part-introduction}

\chapter{Workshop Info}\label{workshop-info}

Welcome to the 2025 Introduction to R Canadian Bioinformatics Workshop webpage!

\section{Pre-work}\label{pre-work}

\href{https://docs.google.com/forms/d/e/1FAIpQLSeJxCyYGuRbbbmOsibx7BnSH3F1SY_IheVu5ira-5uoDMZ_xA/viewform?usp=dialog}{You can find your pre-work here.}

\section{Class Photo}\label{class-photo}

\section{Schedule}\label{schedule}

\chapter{Meet Your Faculty}\label{meet-your-faculty}

\subsubsection{Mohamed Helmy}\label{mohamed-helmy}

Principal Scientist and Adjunct Professor
Vaccine and Infectious Disease Organization (VIDO), University of Saskatchewan
Saskatoon, Saskatchewan, Canada

\href{mailto:mohamed.helmy@usask.ca}{\nolinkurl{mohamed.helmy@usask.ca}}

Mohamed is a Computational Systems Biologist and Principal Scientist leading the Bioinformatics and Systems Biology Lab (BSBL) at the Vaccine and Infectious Disease Organization (VIDO), University of Saskatchewan. He received his MSc and PhD in Computational Systems Biology from Keio University (Tokyo, Japan) and completed his postdoctoral training in bioinformatics at Kyoto University and the University of Toronto. Mohamed's interdisciplinary research profile bridges biology, computer science, and public health.

\subsubsection{Sylvia Li}\label{sylvia-li}

Graduate student
Vaccine and Infectious Disease Organization (VIDO), University of Saskatchewan
Saskatoon, Saskatchewan, Canada

Sylvia is a Computer science MSc student at the University of Saskatchewan, supervised by Dr.~Helmy. She holds dual BSc degrees in Bioinformatics and Computer science. Currently her work focuses on bacterial genomic data.

Data and Compute Setup

\subsubsection{Course data downloads}\label{course-data-downloads}

Coming soon!

\subsubsection{Compute setup}\label{compute-setup}

Coming soon!

\part{Modules}\label{part-modules}

\chapter{Module 1}\label{module-1}

\section{Lecture}\label{lecture}

\subsection{1A}\label{a}

\subsection{1B}\label{b}

\section{Lab 1A}\label{lab-1a}

\subsection{Variables}\label{variables}

Create 2 numeric variables and assign values for each

\begin{Shaded}
\begin{Highlighting}[]
\NormalTok{x }\OtherTok{=} \DecValTok{10}
\NormalTok{y }\OtherTok{=} \DecValTok{6}
\end{Highlighting}
\end{Shaded}

Calculate the sum of them

\begin{Shaded}
\begin{Highlighting}[]
\NormalTok{total }\OtherTok{=}\NormalTok{ x }\SpecialCharTok{+}\NormalTok{ y}
\NormalTok{total}
\end{Highlighting}
\end{Shaded}

\begin{verbatim}
## [1] 16
\end{verbatim}

Calculate the square root of the total

\begin{Shaded}
\begin{Highlighting}[]
\NormalTok{sr }\OtherTok{=} \FunctionTok{sqrt}\NormalTok{(total)}
\NormalTok{sr}
\end{Highlighting}
\end{Shaded}

\begin{verbatim}
## [1] 4
\end{verbatim}

\subsection{Data Structures}\label{data-structures}

Vector

\begin{Shaded}
\begin{Highlighting}[]
\NormalTok{v }\OtherTok{\textless{}{-}} \FunctionTok{c}\NormalTok{(}\DecValTok{1}\NormalTok{,}\DecValTok{2}\NormalTok{,}\DecValTok{3}\NormalTok{,}\DecValTok{4}\NormalTok{)}
\NormalTok{v}
\end{Highlighting}
\end{Shaded}

\begin{verbatim}
## [1] 1 2 3 4
\end{verbatim}

Matrix

\begin{Shaded}
\begin{Highlighting}[]
\NormalTok{m }\OtherTok{\textless{}{-}} \FunctionTok{matrix}\NormalTok{(}\DecValTok{1}\SpecialCharTok{:}\DecValTok{6}\NormalTok{, }\AttributeTok{nrow =} \DecValTok{2}\NormalTok{)}
\NormalTok{m}
\end{Highlighting}
\end{Shaded}

\begin{verbatim}
##      [,1] [,2] [,3]
## [1,]    1    3    5
## [2,]    2    4    6
\end{verbatim}

Dataframe

\begin{Shaded}
\begin{Highlighting}[]
\NormalTok{df }\OtherTok{\textless{}{-}} \FunctionTok{data.frame}\NormalTok{(}\AttributeTok{age=}\FunctionTok{c}\NormalTok{(}\DecValTok{25}\NormalTok{,}\DecValTok{30}\NormalTok{), }\AttributeTok{name=}\FunctionTok{c}\NormalTok{(}\StringTok{"Mo"}\NormalTok{,}\StringTok{"Tom"}\NormalTok{), }\AttributeTok{group=}\FunctionTok{c}\NormalTok{(}\StringTok{"A"}\NormalTok{, }\StringTok{"B"}\NormalTok{))}
\NormalTok{df}
\end{Highlighting}
\end{Shaded}

\begin{verbatim}
##   age name group
## 1  25   Mo     A
## 2  30  Tom     B
\end{verbatim}

List

\begin{Shaded}
\begin{Highlighting}[]
\NormalTok{lst }\OtherTok{\textless{}{-}} \FunctionTok{list}\NormalTok{(}\AttributeTok{numbers=}\NormalTok{v, }\AttributeTok{info=}\NormalTok{df)}
\NormalTok{lst}
\end{Highlighting}
\end{Shaded}

\begin{verbatim}
## $numbers
## [1] 1 2 3 4
## 
## $info
##   age name group
## 1  25   Mo     A
## 2  30  Tom     B
\end{verbatim}

\subsection{Install BioconductoR packages}\label{install-bioconductor-packages}

\begin{Shaded}
\begin{Highlighting}[]
\FunctionTok{install.packages}\NormalTok{(}\StringTok{"BiocManager"}\NormalTok{)}
\FunctionTok{library}\NormalTok{(BiocManager)}
\NormalTok{BiocManager}\SpecialCharTok{::}\FunctionTok{install}\NormalTok{(}\StringTok{"ALL"}\NormalTok{)}
\FunctionTok{library}\NormalTok{(}\StringTok{"ALL"}\NormalTok{)}
\FunctionTok{data}\NormalTok{(ALL)}
\end{Highlighting}
\end{Shaded}

\chapter{Module 2}\label{module-2}

\section{Lecture}\label{lecture-1}

\subsection{2A}\label{a-1}

\subsection{2B}\label{b-1}

\section{Lab}\label{lab}

\chapter{Module 3}\label{module-3}

\section{Lecture}\label{lecture-2}

\section{Lab}\label{lab-1}

\chapter{Module 4}\label{module-4}

\section{Lecture}\label{lecture-3}

\section{Lab}\label{lab-2}

  \bibliography{book.bib,packages.bib}

\end{document}
